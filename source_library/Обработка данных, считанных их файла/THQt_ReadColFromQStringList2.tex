\textbf{Входные параметры:}

QStringListFromFile --- отсюда берем информацию;

     k --- номер столбца, начиная с нуля, который считываем;
	 
     VMHL\_VectorResult --- сюда будем записывать результат считывания столбца из матрицы.

\textbf{Возвращаемое значение:}

Количество элементов в столбце. Как только встречает вместо числа символ «-», то функция считает, что вектор закончился.

\textbf{Пример содержимого QStringListFromFile:}

1	2	6

52	3	96

6.4	7	4

    Второй пример содержимого VMHL\_VectorResult.
	
1	2	6	5

52	3	96	5

-	-	4   2

Считается, что файл правильный, ошибки не проверяются. В строке числа разделяются через табуляцию, а десятичные числа используют точку, а не запятую. Во всех столбцах должно быть одинаковое число элементов. Поэтому, если в одном столбце больше элементов, чем в других, то в остальные столбцы на место недостающих элементов ставится знак «-».