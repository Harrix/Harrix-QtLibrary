\textbf{Входные параметры:}

QStringListFromFile --- отсюда берем информацию;
 
    k --- номер столбца, начиная с нуля, который считываем;
 
    VHQt\_VectorResult --- сюда будем записывать результат считывания столбца из матрицы.
	
\textbf{Входные параметры 2:}
	
	 QStringListFromFile --- отсюда берем информацию;
 
    k --- номер столбца, начиная с нуля, который считываем;
 
    VHQt\_VectorResult --- сюда будем записывать результат считывания столбца из матрицы.

\textbf{Возвращаемое значение:}

Число строк.

\textbf{Пример содержимого QStringListFromFile:}

1	2013.04.05	6

52	2013.02.25	96

6.4	2013.01.15	4

Считается, что файл правильный, ошибки не проверяются. В строке числа разделяются через табуляцию, а десятичные числа используют точку, а не запятую. Во всех столбцах должно быть одинаковое число элементов. Поэтому, если в одном столбце больше элементов, чем в других, то в остальные столбцы на место недостающих элементов ставится знак «-».