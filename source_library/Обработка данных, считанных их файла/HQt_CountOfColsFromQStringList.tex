\textbf{Входные параметры:}

QStringListFromFile - непосредственно сам файл (его содержимое).

\textbf{Возвращаемое значение:}

Число столбцов (по первой строке).

\textbf{Пример содержимого QStringListFromFile:}

1	2.2

2.8	9

Считается, что файл правильный, ошибки не проверяются. В строке числа разделяются через табуляцию, а десятичные числа используют точку, а не запятую. Во всех столбцах должно быть одинаковое число элементов. Поэтому, если в одном столбце больше элементов, чем в других, то в остальные столбцы на место недостающих элементов ставится знак «-».