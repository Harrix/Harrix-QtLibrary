\documentclass[a4paper,12pt]{article}

%%% HarrixLaTeXDocumentTemplate
%%% Версия 1.18
%%% Шаблон документов в LaTeX на русском языке. Данный шаблон применяется в проектах HarrixTestFunctions, MathHarrixLibrary, Standard-Genetic-Algorithm  и др.
%%% https://github.com/Harrix/HarrixLaTeXDocumentTemplate
%%% Шаблон распространяется по лицензии Apache License, Version 2.0.

%%% Поля и разметка страницы %%%
\usepackage{lscape} % Для включения альбомных страниц
\usepackage{geometry} % Для последующего задания полей

%%% Кодировки и шрифты %%%
\usepackage{pscyr} % Нормальные шрифты
\usepackage{cmap} % Улучшенный поиск русских слов в полученном pdf-файле
\usepackage[T2A]{fontenc} % Поддержка русских букв
\usepackage[utf8]{inputenc} % Кодировка utf8
\usepackage[english, russian]{babel} % Языки: русский, английский

%%% Математические пакеты %%%
\usepackage{amsthm,amsfonts,amsmath,amssymb,amscd} % Математические дополнения от AMS
%%% Для жиного курсива в формулах %%%
\usepackage{bm}

%%% Оформление абзацев %%%
\usepackage{indentfirst} % Красная строка
\usepackage{setspace} % Расстояние между строками
\usepackage{enumitem} % Для список обнуление расстояния до абзаца

%%% Цвета %%%
\usepackage[usenames]{color}
\usepackage{color}
\usepackage{colortbl}

%%% Таблицы %%%
\usepackage{longtable} % Длинные таблицы
\usepackage{multirow,makecell,array} % Улучшенное форматирование таблиц

%%% Общее форматирование
\usepackage[singlelinecheck=off,center]{caption} % Многострочные подписи
\usepackage{soul} % Поддержка переносоустойчивых подчёркиваний и зачёркиваний

%%% Библиография %%%
\usepackage{cite}

%%% Гиперссылки %%%
\usepackage{hyperref}

%%% Изображения %%%
\usepackage{graphicx} % Подключаем пакет работы с графикой
\usepackage{epstopdf}
\usepackage{subcaption}

%%% Отображение кода %%%
\usepackage{xcolor}
\usepackage{listings}
\usepackage{caption}

%%% Псевдокоды %%%
\usepackage{algorithm} 
\usepackage{algpseudocode}

%%% Рисование графиков %%%
\usepackage{pgfplots}

%%% HarrixLaTeXDocumentTemplate
%%% Версия 1.20
%%% Шаблон документов в LaTeX на русском языке. Данный шаблон применяется в проектах HarrixTestFunctions, MathHarrixLibrary, Standard-Genetic-Algorithm  и др.
%%% https://github.com/Harrix/HarrixLaTeXDocumentTemplate
%%% Шаблон распространяется по лицензии Apache License, Version 2.0.

%%% Макет страницы %%%
\geometry{a4paper,top=2cm,bottom=2cm,left=2cm,right=1cm}

%%% Кодировки и шрифты %%%
%\renewcommand{\rmdefault}{ftm} % Включаем Times New Roman

%%% Выравнивание и переносы %%%
\sloppy
\clubpenalty=10000
\widowpenalty=10000

%%% Библиография %%%
\makeatletter
\bibliographystyle{utf8gost705u} % Оформляем библиографию в соответствии с ГОСТ 7.0.5
\renewcommand{\@biblabel}[1]{#1.} % Заменяем библиографию с квадратных скобок на точку:
\makeatother

%%% Изображения %%%
\graphicspath{{images/}} % Пути к изображениям
% Поменять двоеточние на точку в подписях к рисунку
\RequirePackage{caption}
\DeclareCaptionLabelSeparator{defffis}{. }
\captionsetup{justification=centering,labelsep=defffis}

%%% Цвета %%%
% Цвета для кода
\definecolor{string}{HTML}{B40000} % цвет строк в коде
\definecolor{comment}{HTML}{008000} % цвет комментариев в коде
\definecolor{keyword}{HTML}{1A00FF} % цвет ключевых слов в коде
\definecolor{morecomment}{HTML}{8000FF} % цвет include и других элементов в коде
\definecolor{сaptiontext}{HTML}{FFFFFF} % цвет текста заголовка в коде
\definecolor{сaptionbk}{HTML}{999999} % цвет фона заголовка в коде
\definecolor{bk}{HTML}{FFFFFF} % цвет фона в коде
\definecolor{frame}{HTML}{999999} % цвет рамки в коде
\definecolor{brackets}{HTML}{B40000} % цвет скобок в коде
% Цвета для гиперссылок
\definecolor{linkcolor}{HTML}{799B03} % цвет ссылок
\definecolor{urlcolor}{HTML}{799B03} % цвет гиперссылок
\definecolor{citecolor}{HTML}{799B03} % цвет гиперссылок
\definecolor{gray}{rgb}{0.4,0.4,0.4}
\definecolor{tableheadcolor}{HTML}{E5E5E5} % цвет шапки в таблицах
\definecolor{darkblue}{rgb}{0.0,0.0,0.6}
% Цвета для графиков
\definecolor{plotcoordinate}{HTML}{88969C}% цвет точек на координатых осях (минимум и максимум)
\definecolor{plotgrid}{HTML}{ECECEC} % цвет сетки
\definecolor{plotmain}{HTML}{97BBCD} % цвет основного графика
\definecolor{plotsecond}{HTML}{FF0000} % цвет второго графика, если графика только два
\definecolor{plotsecondgray}{HTML}{CCCCCC} % цвет второго графика, если графика только два. В сером виде.
\definecolor{darkgreen}{HTML}{799B03} % цвет темно-зеленого

%%% Отображение кода %%%
% Настройки отображения кода
\lstset{
language=C++, % Язык кода по умолчанию
morekeywords={*,...}, % если хотите добавить ключевые слова, то добавляйте
% Цвета
keywordstyle=\color{keyword}\ttfamily\bfseries,
%stringstyle=\color{string}\ttfamily,
stringstyle=\ttfamily\color{red!50!brown},
commentstyle=\color{comment}\ttfamily\itshape,
morecomment=[l][\color{morecomment}]{\#}, 
% Настройки отображения     
breaklines=true, % Перенос длинных строк
basicstyle=\ttfamily\footnotesize, % Шрифт для отображения кода
backgroundcolor=\color{bk}, % Цвет фона кода
frame=lrb,xleftmargin=\fboxsep,xrightmargin=-\fboxsep, % Рамка, подогнанная к заголовку
rulecolor=\color{frame}, % Цвет рамки
tabsize=3, % Размер табуляции в пробелах
% Настройка отображения номеров строк. Если не нужно, то удалите весь блок
%numbers=left, % Слева отображаются номера строк
%stepnumber=1, % Каждую строку нумеровать
%numbersep=5pt, % Отступ от кода 
%numberstyle=\small\color{black}, % Стиль написания номеров строк
% Для отображения русского языка
extendedchars=true,
literate={Ö}{{\"O}}1
  {Ä}{{\"A}}1
  {Ü}{{\"U}}1
  {ß}{{\ss}}1
  {ü}{{\"u}}1
  {ä}{{\"a}}1
  {ö}{{\"o}}1
  {~}{{\textasciitilde}}1
  {а}{{\selectfont\char224}}1
  {б}{{\selectfont\char225}}1
  {в}{{\selectfont\char226}}1
  {г}{{\selectfont\char227}}1
  {д}{{\selectfont\char228}}1
  {е}{{\selectfont\char229}}1
  {ё}{{\"e}}1
  {ж}{{\selectfont\char230}}1
  {з}{{\selectfont\char231}}1
  {и}{{\selectfont\char232}}1
  {й}{{\selectfont\char233}}1
  {к}{{\selectfont\char234}}1
  {л}{{\selectfont\char235}}1
  {м}{{\selectfont\char236}}1
  {н}{{\selectfont\char237}}1
  {о}{{\selectfont\char238}}1
  {п}{{\selectfont\char239}}1
  {р}{{\selectfont\char240}}1
  {с}{{\selectfont\char241}}1
  {т}{{\selectfont\char242}}1
  {у}{{\selectfont\char243}}1
  {ф}{{\selectfont\char244}}1
  {х}{{\selectfont\char245}}1
  {ц}{{\selectfont\char246}}1
  {ч}{{\selectfont\char247}}1
  {ш}{{\selectfont\char248}}1
  {щ}{{\selectfont\char249}}1
  {ъ}{{\selectfont\char250}}1
  {ы}{{\selectfont\char251}}1
  {ь}{{\selectfont\char252}}1
  {э}{{\selectfont\char253}}1
  {ю}{{\selectfont\char254}}1
  {я}{{\selectfont\char255}}1
  {А}{{\selectfont\char192}}1
  {Б}{{\selectfont\char193}}1
  {В}{{\selectfont\char194}}1
  {Г}{{\selectfont\char195}}1
  {Д}{{\selectfont\char196}}1
  {Е}{{\selectfont\char197}}1
  {Ё}{{\"E}}1
  {Ж}{{\selectfont\char198}}1
  {З}{{\selectfont\char199}}1
  {И}{{\selectfont\char200}}1
  {Й}{{\selectfont\char201}}1
  {К}{{\selectfont\char202}}1
  {Л}{{\selectfont\char203}}1
  {М}{{\selectfont\char204}}1
  {Н}{{\selectfont\char205}}1
  {О}{{\selectfont\char206}}1
  {П}{{\selectfont\char207}}1
  {Р}{{\selectfont\char208}}1
  {С}{{\selectfont\char209}}1
  {Т}{{\selectfont\char210}}1
  {У}{{\selectfont\char211}}1
  {Ф}{{\selectfont\char212}}1
  {Х}{{\selectfont\char213}}1
  {Ц}{{\selectfont\char214}}1
  {Ч}{{\selectfont\char215}}1
  {Ш}{{\selectfont\char216}}1
  {Щ}{{\selectfont\char217}}1
  {Ъ}{{\selectfont\char218}}1
  {Ы}{{\selectfont\char219}}1
  {Ь}{{\selectfont\char220}}1
  {Э}{{\selectfont\char221}}1
  {Ю}{{\selectfont\char222}}1
  {Я}{{\selectfont\char223}}1
  {і}{{\selectfont\char105}}1
  {ї}{{\selectfont\char168}}1
  {є}{{\selectfont\char185}}1
  {ґ}{{\selectfont\char160}}1
  {І}{{\selectfont\char73}}1
  {Ї}{{\selectfont\char136}}1
  {Є}{{\selectfont\char153}}1
  {Ґ}{{\selectfont\char128}}1
  {\{}{{{\color{brackets}\{}}}1 % Цвет скобок {
  {\}}{{{\color{brackets}\}}}}1 % Цвет скобок }
}
% Для настройки заголовка кода
\DeclareCaptionFont{white}{\color{сaptiontext}}
\DeclareCaptionFormat{listing}{\parbox{\linewidth}{\colorbox{сaptionbk}{\parbox{\linewidth}{#1#2#3}}\vskip-4pt}}
\captionsetup[lstlisting]{format=listing,labelfont=white,textfont=white}
\renewcommand{\lstlistingname}{Код} % Переименование Listings в нужное именование структуры
% Для отображения кода формата xml
\lstdefinelanguage{XML}
{
  morestring=[s]{"}{"},
  morecomment=[s]{?}{?},
  morecomment=[s]{!--}{--},
  commentstyle=\color{comment},
  moredelim=[s][\color{black}]{>}{<},
  moredelim=[s][\color{red}]{\ }{=},
  stringstyle=\color{string},
  identifierstyle=\color{keyword}
}

%%% Гиперссылки %%%
\hypersetup{pdfstartview=FitH,  linkcolor=linkcolor,urlcolor=urlcolor,citecolor=citecolor, colorlinks=true}

%%%  Оформление абзацев %%%
\setlength{\parskip}{0.3cm} % отступы между абзацами
% оформление списков
\setlist{leftmargin=1.5cm,topsep=0pt}

%%% Псевдокоды %%%
% Добавляем свои блоки
\makeatletter
\algblock[ALGORITHMBLOCK]{BeginAlgorithm}{EndAlgorithm}
\algblock[BLOCK]{BeginBlock}{EndBlock}
\makeatother

% Нумерация блоков
\usepackage{caption}% http://ctan.org/pkg/caption
\captionsetup[ruled]{labelsep=period}
\makeatletter
\@addtoreset{algorithm}{chapter}% algorithm counter resets every chapter
\makeatother
\renewcommand{\thealgorithm}{\thechapter.\arabic{algorithm}}% Algorithm # is <chapter>.<algorithm>

%%% Формулы %%%
%Дублирование символа при переносе
\newcommand{\hmm}[1]{#1\nobreak\discretionary{}{\hbox{\ensuremath{#1}}}{}}

%%% Таблицы %%%
% Раздвигаем в таблице без границ отступы между строками в новой команде
\newenvironment{tabularwide}%
{\setlength{\extrarowheight}{0.3cm}\begin{tabular}{  p{\dimexpr 0.45\linewidth-2\tabcolsep} p{\dimexpr 0.55\linewidth-2\tabcolsep}  }}  {\end{tabular}}
\newenvironment{tabularwide08}%
{\setlength{\extrarowheight}{0.3cm}\begin{tabular}{  p{\dimexpr 0.8\linewidth-2\tabcolsep} p{\dimexpr 0.2\linewidth-2\tabcolsep}  }}  {\end{tabular}}

% Многострочная ячейка в таблице
\newcommand{\specialcell}[2][c]{%
  {\renewcommand{\arraystretch}{1}\begin{tabular}[t]{@{}l@{}}#2\end{tabular}}}

% Многострочная ячейка, где текст не может выйти за границы
\newcolumntype{P}[1]{>{\raggedright\arraybackslash}p{#1}}
\newcommand{\specialcelltwoin}[2][c]{%
  {\renewcommand{\arraystretch}{1}\begin{tabular}[t]{@{}P{1.98in}@{}}#2\end{tabular}}}
  
% Команда для переворачивания текста в ячейке таблицы на 90 градусов
\newcommand*\rot{\rotatebox{90}}
  
%%% Абзацы %%%
% Отсупы между строками
\singlespacing

%%% Рисование графиков %%%
\pgfplotsset{
every axis legend/.append style={at={(0.5,-0.13)},anchor=north,legend cell align=left},
tick label style={font=\tiny\scriptsize},
label style={font=\scriptsize},
legend style={font=\scriptsize},
grid=both,
minor tick num=2,
major grid style={plotgrid},
minor grid style={plotgrid},
axis lines=left,
legend style={draw=none},
/pgf/number format/.cd,
1000 sep={}
}
% Карта цвета для трехмерных графиков в стиле графиков Mathcad
\pgfplotsset{
/pgfplots/colormap={mathcad}{rgb255(0cm)=(76,0,128) rgb255(2cm)=(0,14,147) rgb255(4cm)=(0,173,171) rgb255(6cm)=(32,205,0) rgb255(8cm)=(229,222,0) rgb255(10cm)=(255,13,0)}
}
% Карта цвета для трехмерных графиков в стиле графиков Matlab
\pgfplotsset{
/pgfplots/colormap={matlab}{rgb255(0cm)=(0,0,128) rgb255(1cm)=(0,0,255) rgb255(3cm)=(0,255,255) rgb255(5cm)=(255,255,0) rgb255(7cm)=(255,0,0) rgb255(8cm)=(128,0,0)}
}

%%% Разное %%%
% Галочки для отмечания в тескте вариантов как OK
\def\checkmark{\tikz\fill[black,scale=0.3](0,.35) -- (.25,0) -- (1,.7) -- (.25,.15) -- cycle;}
\def\checkmarkgreen{\tikz\fill[darkgreen,scale=0.3](0,.35) -- (.25,0) -- (1,.7) -- (.25,.15) -- cycle;} 
\def\checkmarkred{\tikz\fill[red,scale=0.3](0,.35) -- (.25,0) -- (1,.7) -- (.25,.15) -- cycle;}
\def\checkmarkbig{\tikz\fill[black,scale=0.5](0,.35) -- (.25,0) -- (1,.7) -- (.25,.15) -- cycle;}
\def\checkmarkbiggreen{\tikz\fill[darkgreen,scale=0.5](0,.35) -- (.25,0) -- (1,.7) -- (.25,.15) -- cycle;} 
\def\checkmarkbigred{\tikz\fill[red,scale=0.5](0,.35) -- (.25,0) -- (1,.7) -- (.25,.15) -- cycle;} 

\title{HarrixQtLibrary v.3.27}
\author{А.\,Б. Сергиенко}
\date{\today}


\begin{document}

%%% HarrixLaTeXDocumentTemplate
%%% Версия 1.21
%%% Шаблон документов в LaTeX на русском языке. Данный шаблон применяется в проектах HarrixTestFunctions, MathHarrixLibrary, Standard-Genetic-Algorithm  и др.
%%% https://github.com/Harrix/HarrixLaTeXDocumentTemplate
%%% Шаблон распространяется по лицензии Apache License, Version 2.0.

%%% Именования %%%
\renewcommand{\abstractname}{Аннотация}
\renewcommand{\alsoname}{см. также}
\renewcommand{\appendixname}{Приложение} % (ГОСТ Р 7.0.11-2011, 5.7)
\renewcommand{\bibname}{Список литературы} % (ГОСТ Р 7.0.11-2011, 4)
\renewcommand{\ccname}{исх.}
\renewcommand{\chaptername}{Глава}
\renewcommand{\contentsname}{Оглавление} % (ГОСТ Р 7.0.11-2011, 4)
\renewcommand{\enclname}{вкл.}
\renewcommand{\figurename}{Рисунок} % (ГОСТ Р 7.0.11-2011, 5.3.9)
\renewcommand{\headtoname}{вх.}
\renewcommand{\indexname}{Предметный указатель}
\renewcommand{\listfigurename}{Список рисунков}
\renewcommand{\listtablename}{Список таблиц}
\renewcommand{\pagename}{Стр.}
\renewcommand{\partname}{Часть}
\renewcommand{\refname}{Список литературы} % (ГОСТ Р 7.0.11-2011, 4)
\renewcommand{\seename}{см.}
\renewcommand{\tablename}{Таблица} % (ГОСТ Р 7.0.11-2011, 5.3.10)

%%% Псевдокоды %%%
% Перевод данных об алгоритмах
\renewcommand{\listalgorithmname}{Список алгоритмов}
\floatname{algorithm}{Алгоритм}

% Перевод команд псевдокода
\algrenewcommand\algorithmicwhile{\textbf{До тех пока}}
\algrenewcommand\algorithmicdo{\textbf{выполнять}}
\algrenewcommand\algorithmicrepeat{\textbf{Повторять}}
\algrenewcommand\algorithmicuntil{\textbf{Пока выполняется}}
\algrenewcommand\algorithmicend{\textbf{Конец}}
\algrenewcommand\algorithmicif{\textbf{Если}}
\algrenewcommand\algorithmicelse{\textbf{иначе}}
\algrenewcommand\algorithmicthen{\textbf{тогда}}
\algrenewcommand\algorithmicfor{\textbf{Цикл. }}
\algrenewcommand\algorithmicforall{\textbf{Выполнить для всех}}
\algrenewcommand\algorithmicfunction{\textbf{Функция}}
\algrenewcommand\algorithmicprocedure{\textbf{Процедура}}
\algrenewcommand\algorithmicloop{\textbf{Зациклить}}
\algrenewcommand\algorithmicrequire{\textbf{Условия:}}
\algrenewcommand\algorithmicensure{\textbf{Обеспечивающие условия:}}
\algrenewcommand\algorithmicreturn{\textbf{Возвратить}}
\algrenewtext{EndWhile}{\textbf{Конец цикла}}
\algrenewtext{EndLoop}{\textbf{Конец зацикливания}}
\algrenewtext{EndFor}{\textbf{Конец цикла}}
\algrenewtext{EndFunction}{\textbf{Конец функции}}
\algrenewtext{EndProcedure}{\textbf{Конец процедуры}}
\algrenewtext{EndIf}{\textbf{Конец условия}}
\algrenewtext{EndFor}{\textbf{Конец цикла}}
\algrenewtext{BeginAlgorithm}{\textbf{Начало алгоритма}}
\algrenewtext{EndAlgorithm}{\textbf{Конец алгоритма}}
\algrenewtext{BeginBlock}{\textbf{Начало блока. }}
\algrenewtext{EndBlock}{\textbf{Конец блока}}
\algrenewtext{ElsIf}{\textbf{иначе если }}

\maketitle

\begin{abstract}
Библиотека HarrixQtLibrary --- сборник функций для Qt.
\end{abstract}

\tableofcontents

\newpage

\section{Введение}

Библиотека HarrixQtLibrary --- это сборник функций для Qt.

Последнюю версию документа можно найти по адресу:

\href{https://github.com/Harrix/HarrixQtLibrary}{https://github.com/Harrix/HarrixQtLibrary}

Об установке библиотеки можно прочитать тут:

\href{http://blog.harrix.org/?p=1186}{http://blog.harrix.org/?p=1186}

С автором можно связаться по адресу \href{mailto:sergienkoanton@mail.ru}{sergienkoanton@mail.ru} или  \href{http://vk.com/harrix}{http://vk.com/harrix}.

Сайт автора, где публикуются последние новости: \href{http://blog.harrix.org/}{http://blog.harrix.org/}, а проекты располагаются по адресу \href{http://harrix.org/}{http://harrix.org/}.

\newpage
\section{Список функций}\label{section_listfunctions}
\textbf{Обработка данных, считанных их файла}
\begin{enumerate}

\item \textbf{\hyperref[HQt_CountOfColsFromQStringList]{HQt\_CountOfColsFromQStringList}} --- Функция подсчитывает сколько столбцов в текстовом файле, который скопировали в QStringListFromFile.

\item \textbf{\hyperref[HQt_CountOfRowsFromQStringList]{HQt\_CountOfRowsFromQStringList}} --- Функция подсчитывает сколько строк в текстовом файле, который скопировали в QStringListFromFile. Функция подсчитывает сколько строк в k столбце из текстового файла, который скопировали в QStringListFromFile. Функция подсчитывает сколько строк в каждом столбце из текстового файла с матрицей, который скопировали в QStringListFromFile.

\item \textbf{\hyperref[THQt_ReadColFromQStringList]{THQt\_ReadColFromQStringList}} --- Функция считывает данные какого-то k столбца с датами из QStringList в виде матрицы. Функция считывает данные какого-то k столбца с датами из QStringList в виде матрицы. Для строк.

\item \textbf{\hyperref[THQt_ReadColFromQStringList2]{THQt\_ReadColFromQStringList2}} --- Функция считывает данные какого-то k столбца из QStringList в виде матрицы.

\item \textbf{\hyperref[THQt_ReadMatrixFromQStringList]{THQt\_ReadMatrixFromQStringList}} --- Функция считывает данные из QStringList в матрицу.

\item \textbf{\hyperref[THQt_ReadTwoVectorFromQStringList]{THQt\_ReadTwoVectorFromQStringList}} --- Функция считывает данные из QStringList в два вектора.

\item \textbf{\hyperref[THQt_ReadVectorFromQStringList]{THQt\_ReadVectorFromQStringList}} --- Функция считывает данные из QStringList в вектор.

\end{enumerate}

\textbf{Получение строк-выводов по данным}
\begin{enumerate}

\item \textbf{\hyperref[HQt_BoolToWord]{HQt\_BoolToWord}} --- Функция переводит переменную, принимающую значения "0" и "1" в слова.

\item \textbf{\hyperref[HQt_RandomString]{HQt\_RandomString}} --- Функция генерирует случайную строку из английских больших и малых букв.

\item \textbf{\hyperref[HQt_TryingReduceString]{HQt\_TryingReduceString}} --- Функция старается сократить строку длиной больше MaxSize символов, сокращая слова. Функция старается сократить строку длиной больше 40 символов, сокращая слова.

\item \textbf{\hyperref[HQt_UniqueName]{HQt\_UniqueName}} --- Функция возвращает уникальную строку, которую можно использовать как некий идентификатор. Собирается из "HQt\_" + текущее время или из BeginString+"\_" + текущее время.

\item \textbf{\hyperref[HQt_UniqueNameOnlyNumbers]{HQt\_UniqueNameOnlyNumbers}} --- Функция возвращает уникальную строку, которую можно использовать как некий идентификатор. В строке только цифры. Собирается из текущего времени.

\item \textbf{\hyperref[HQt_WriteTime]{HQt\_WriteTime}} --- Функция переводит миллисекунды в строку с описанием сколько это минут, секунд и др.

\end{enumerate}

\textbf{Работа с алфавитом и переносами}
\begin{enumerate}

\item \textbf{\hyperref[HQt_BreakdownOfTextWithWordWrap]{HQt\_BreakdownOfTextWithWordWrap}} --- Функция разбивает текст на строки длиной не более length. Если может, то ставит переносы.

\item \textbf{\hyperref[HQt_CheckIntolerablePunctuation]{HQt\_CheckIntolerablePunctuation}} --- Является ли символом знаком пунктуации, который нельзя переносить.

\item \textbf{\hyperref[HQt_CheckLetterFromWord]{HQt\_CheckLetterFromWord}} --- Является ли буква символом из слова. Считается, что это или латинские буквы, или русские, или цифры или нижнее подчеркивание. Плюc некоторые знаки препинания, так как их нельзя переносить.

\item \textbf{\hyperref[HQt_CheckRus]{HQt\_CheckRus}} --- Функция проверяет наличие русских букв в строке.

\item \textbf{\hyperref[HQt_CutToWords]{HQt\_CutToWords}} --- Функция разбивает строку на слова и те, части, между которыми они находятся. Слова с дефисом считаются за несколько слов.

\item \textbf{\hyperref[HQt_CutToWordsWithWordWrap]{HQt\_CutToWordsWithWordWrap}} --- Функция разбивает строку на слова и те, части, между которыми они находятся. А русские и английские слова разделяет по переносам. Слова с дефисом считаются за несколько слов.

\item \textbf{\hyperref[HQt_GetTypeCharEng]{HQt\_GetTypeCharEng}} --- Функция выдает тип вводимого QString (считая, что это буква английская). Нужно для алгоритма переноса строк П.Хpистова в модификации Дымченко и Ваpсанофьева.

\item \textbf{\hyperref[HQt_GetTypeCharRus]{HQt\_GetTypeCharRus}} --- Функция выдает тип вводимого QString (считая, что это буква). Нужно для алгоритма переноса строк П.Хpистова в модификации Дымченко и Ваpсанофьева.

\end{enumerate}

\textbf{Работа с датами}
\begin{enumerate}

\item \textbf{\hyperref[HQt_DaysBetweenDates]{HQt\_DaysBetweenDates}} --- Функция определяет сколько дней между двумя датами.

\end{enumerate}

\textbf{Работа с файлами и папками}
\begin{enumerate}

\item \textbf{\hyperref[HQt_CopyFile]{HQt\_CopyFile}} --- Функция копирует файл filename в папку dir. Функция копирует файл filename в папку dir, с возможностью перезаписи (в функции-перегрузке).

\item \textbf{\hyperref[HQt_DirDelete]{HQt\_DirDelete}} --- Функция удаляет директорию и всё ее содержимое.

\item \textbf{\hyperref[HQt_DirExists]{HQt\_DirExists}} --- Функция проверяет существование директории.

\item \textbf{\hyperref[HQt_DirMake]{HQt\_DirMake}} --- Функция проверяет существование директории, и если ее нет, то создает.

\item \textbf{\hyperref[HQt_FileExists]{HQt\_FileExists}} --- Функция проверяет существование файла.

\item \textbf{\hyperref[HQt_GetExpFromFilename]{HQt\_GetExpFromFilename}} --- Функция получает расширение файла по его имени.

\item \textbf{\hyperref[HQt_GetFilenameFromFullFilename]{HQt\_GetFilenameFromFullFilename}} --- Функция получает имя файла по полному пути.

\item \textbf{\hyperref[HQt_GetNameFromFilename]{HQt\_GetNameFromFilename}} --- Функция получает имя файла без расширения по его имени.

\item \textbf{\hyperref[HQt_ListDirsInDir]{HQt\_ListDirsInDir}} --- Функция считывает список директорий в директории в QString.

\item \textbf{\hyperref[HQt_ListDirsInDirQStringList]{HQt\_ListDirsInDirQStringList}} --- Функция считывает список директорий в директории в QStringList..

\item \textbf{\hyperref[HQt_ListFilesInDir]{HQt\_ListFilesInDir}} --- Функция считывает список файлов (включая скрытые) в директории в QString.

\item \textbf{\hyperref[HQt_ListFilesInDirQStringList]{HQt\_ListFilesInDirQStringList}} --- Функция считывает список файлов (включая скрытые) в директории в QStringList.

\item \textbf{\hyperref[HQt_ReadFile]{HQt\_ReadFile}} --- Функция считывает текстовой файл в QString.

\item \textbf{\hyperref[HQt_ReadFileToQStringList]{HQt\_ReadFileToQStringList}} --- Функция считывает текстовой файл в QStringList.

\item \textbf{\hyperref[HQt_RenameFile]{HQt\_RenameFile}} --- Функция переименовывает файл filename в newfilename.

\item \textbf{\hyperref[HQt_SaveFile]{HQt\_SaveFile}} --- Функция сохраняет QString в текстовой файл.

\end{enumerate}

\textbf{Работа с цветом}
\begin{enumerate}

\item \textbf{\hyperref[THQt_AlphaBlendingColorToColor]{THQt\_AlphaBlendingColorToColor}} --- Функция накладывает сверху на цвет другой цвет с определенной прозрачностью.

\item \textbf{\hyperref[THQt_ColorFromGradient]{THQt\_ColorFromGradient}} --- Функция выдает код RGB из градиента от одного цвета FirstRGB к другому цвету SecondRGB согласно позиции от 0 до 1.

\item \textbf{\hyperref[THQt_GiveRainbowColorRGB]{THQt\_GiveRainbowColorRGB}} --- Функция выдает код RGB из градиента радуги для любой позиции от 0 до 1 из этого градиента.

\item \textbf{\hyperref[THQt_RGBStringToThreeNumbers]{THQt\_RGBStringToThreeNumbers}} --- Функция переводит строку RGB типа \#25ffb5 в три числа от 0 до 255, которые кодируют  цвета.

\item \textbf{\hyperref[THQt_ThreeNumbersToRGBString]{THQt\_ThreeNumbersToRGBString}} --- Функция переводит три числа в строку RGB типа \#25ffb5, как в Photoshop или HTML.

\end{enumerate}

\textbf{Работа со строками и списками строк}
\begin{enumerate}

\item \textbf{\hyperref[HQt_AddUniqueQStringInQStringList]{HQt\_AddUniqueQStringInQStringList}} --- Функция добавляет в QStringList строку QString. Но если такая строка уже присутствует, то добавление не происходит.

\item \textbf{\hyperref[HQt_IsNumeric]{HQt\_IsNumeric}} --- Функция проверяет - является ли строка числом.

\item \textbf{\hyperref[HQt_MaxCountOfQStringList]{HQt\_MaxCountOfQStringList}} --- Функция выдает длину макимальной по длине строки в QStringList.

\item \textbf{\hyperref[HQt_NaturalCompareTwoQStrings]{HQt\_NaturalCompareTwoQStrings}} --- Функция сравнивает две строки и определяет какая строчка идет первой. Служебная функция для сортировки строк в обычном стиле, когда строки: z1, z10, z2 сортируются как z1, z2, z10.

\item \textbf{\hyperref[HQt_NaturalSortingQStringList]{HQt\_NaturalSortingQStringList}} --- Функция сортировки строк в сортировки строк QStringList в натуральном виде, например, строки: z1, z10, z2 сортируются как z1, z2, z10.

\item \textbf{\hyperref[HQt_QStringListToQString]{HQt\_QStringListToQString}} --- Функция переводит QStringList в QString.

\item \textbf{\hyperref[HQt_QStringToNumber]{HQt\_QStringToNumber}} --- Функция выводит строку x в число.

\item \textbf{\hyperref[HQt_QStringToQStringList]{HQt\_QStringToQStringList}} --- Функция переводит QString в QStringList.

\item \textbf{\hyperref[HQt_SearchQStringInQStringList]{HQt\_SearchQStringInQStringList}} --- Функция ищет в QStringList строку QString (номер первого вхождения).

\item \textbf{\hyperref[HQt_StringForLaTeX]{HQt\_StringForLaTeX}} --- Функция обрабатывает строку String так, чтобы она подходила для публикации в LaTeX.

\item \textbf{\hyperref[HQt_StringToLabelForLaTeX]{HQt\_StringToLabelForLaTeX}} --- Функция обрабатывает строку String так, чтобы она подходила для публикации в LaTeX в виде label.

\item \textbf{\hyperref[HQt_TextAfterEqualSign]{HQt\_TextAfterEqualSign}} --- Функция возвращает текст строки после первого знака =.

\item \textbf{\hyperref[HQt_TextBeforeEqualSign]{HQt\_TextBeforeEqualSign}} --- Функция возвращает текст строки до первого знака =.

\item \textbf{\hyperref[THQt_VectorToQStringList]{THQt\_VectorToQStringList}} --- Функция переводит вектор чисел в QStringList.

\end{enumerate}

\textbf{Служебные функции}
\begin{enumerate}

\item \textbf{\hyperref[HQt_Delay]{HQt\_Delay}} --- Функция делает задержку в MSecs миллисекунд.

\end{enumerate}


\newpage
\section{Функции}
\subsection{Обработка данных, считанных их файла}

\subsubsection{HQt\_CountOfColsFromQStringList}\label{HQt_CountOfColsFromQStringList}

Функция подсчитывает сколько столбцов в текстовом файле, который скопировали в QStringListFromFile.


\begin{lstlisting}[label=code_syntax_HQt_CountOfColsFromQStringList,caption=Синтаксис]
int HQt_CountOfColsFromQStringList(QStringList QStringListFromFile);
\end{lstlisting}

\textbf{Входные параметры:}

QStringListFromFile - непосредственно сам файл (его содержимое).

\textbf{Возвращаемое значение:}

Число столбцов (по первой строке).

\textbf{Пример содержимого QStringListFromFile:}

1	2.2

2.8	9

Считается, что файл правильный, ошибки не проверяются. В строке числа разделяются через табуляцию, а десятичные числа используют точку, а не запятую. Во всех столбцах должно быть одинаковое число элементов. Поэтому, если в одном столбце больше элементов, чем в других, то в остальные столбцы на место недостающих элементов ставится знак «-».


\subsubsection{HQt\_CountOfRowsFromQStringList}\label{HQt_CountOfRowsFromQStringList}

Функция подсчитывает сколько строк в текстовом файле, который скопировали в QStringListFromFile. Функция подсчитывает сколько строк в k столбце из текстового файла, который скопировали в QStringListFromFile. Функция подсчитывает сколько строк в каждом столбце из текстового файла с матрицей, который скопировали в QStringListFromFile.


\begin{lstlisting}[label=code_syntax_HQt_CountOfRowsFromQStringList,caption=Синтаксис]
int HQt_CountOfRowsFromQStringList(QStringList QStringListFromFile);
int HQt_CountOfRowsFromQStringList(QStringList QStringListFromFile, int k);
int HQt_CountOfRowsFromQStringList(QStringList QStringListFromFile, int *VMHL_ResultVector);
\end{lstlisting}

\textbf{Входные параметры:}

QStringListFromFile --- непосредственно сам файл (его содержимое).

k --- номер столбца (необзяательно).

 VMHL\_ResultVector --- сюда количество строк заносится (необязательно).

\textbf{Возвращаемое значение:}

Число строк.

\textbf{Пример содержимого QStringListFromFile:}

1	2.2

2.8	9

Считается, что файл правильный, ошибки не проверяются. В строке числа разделяются через табуляцию, а десятичные числа используют точку, а не запятую. Во всех столбцах должно быть одинаковое число элементов. Поэтому, если в одном столбце больше элементов, чем в других, то в остальные столбцы на место недостающих элементов ставится знак «-».


\subsubsection{THQt\_ReadColFromQStringList}\label{THQt_ReadColFromQStringList}

Функция считывает данные какого-то k столбца с датами из QStringList в виде матрицы. Функция считывает данные какого-то k столбца с датами из QStringList в виде матрицы. Для строк.


\begin{lstlisting}[label=code_syntax_THQt_ReadColFromQStringList,caption=Синтаксис]
void THQt_ReadColFromQStringList(QStringList QStringListFromFile, int k, QDate *VMHL_VectorResult);
void THQt_ReadColFromQStringList(QStringList QStringListFromFile, int k, QString *VMHL_VectorResult);
\end{lstlisting}

\textbf{Входные параметры:}

QStringListFromFile --- отсюда берем информацию;
 
    k --- номер столбца, начиная с нуля, который считываем;
 
    VMHL\_VectorResult --- сюда будем записывать результат считывания столбца из матрицы.
	
\textbf{Входные параметры 2:}
	
	 QStringListFromFile --- отсюда берем информацию;
 
    k --- номер столбца, начиная с нуля, который считываем;
 
    VMHL\_VectorResult --- сюда будем записывать результат считывания столбца из матрицы.

\textbf{Возвращаемое значение:}

Число строк.

\textbf{Пример содержимого QStringListFromFile:}

1	2013.04.05	6

52	2013.02.25	96

6.4	2013.01.15	4

Считается, что файл правильный, ошибки не проверяются. В строке числа разделяются через табуляцию, а десятичные числа используют точку, а не запятую. Во всех столбцах должно быть одинаковое число элементов. Поэтому, если в одном столбце больше элементов, чем в других, то в остальные столбцы на место недостающих элементов ставится знак «-».


\subsubsection{THQt\_ReadColFromQStringList2}\label{THQt_ReadColFromQStringList2}

Функция считывает данные какого-то k столбца из QStringList в виде матрицы.


\begin{lstlisting}[label=code_syntax_THQt_ReadColFromQStringList2,caption=Синтаксис]
template <class T> void THQt_ReadColFromQStringList(QStringList QStringListFromFile, int k, T *VMHL_VectorResult);
\end{lstlisting}

\textbf{Входные параметры:}

QStringListFromFile --- отсюда берем информацию;

     k --- номер столбца, начиная с нуля, который считываем;
	 
     VMHL\_VectorResult --- сюда будем записывать результат считывания столбца из матрицы.

\textbf{Возвращаемое значение:}

Количество элементов в столбце. Как только встречает вместо числа символ «-», то функция считает, что вектор закончился.

\textbf{Пример содержимого QStringListFromFile:}

1	2	6

52	3	96

6.4	7	4

    Второй пример содержимого VMHL\_VectorResult.
	
1	2	6	5

52	3	96	5

-	-	4   2

Считается, что файл правильный, ошибки не проверяются. В строке числа разделяются через табуляцию, а десятичные числа используют точку, а не запятую. Во всех столбцах должно быть одинаковое число элементов. Поэтому, если в одном столбце больше элементов, чем в других, то в остальные столбцы на место недостающих элементов ставится знак «-».


\begin{lstlisting}[label=code_use_THQt_ReadColFromQStringList2,caption=Пример использования]
QString DS=QDir::separator();
QString path=QGuiApplication::applicationDirPath()+DS;//путь к папке

QStringList List = HQt_ReadFileToQStringList(path+"5.txt");
int N;
N=HQt_CountOfRowsFromQStringList(List,k);

double *X;
X=new double[N];

int k=2;//номер столбца

THQt_ReadColFromQStringList(List, k, X);

delete [] X;
\end{lstlisting}

\subsubsection{THQt\_ReadMatrixFromQStringList}\label{THQt_ReadMatrixFromQStringList}

Функция считывает данные из QStringList в матрицу.


\begin{lstlisting}[label=code_syntax_THQt_ReadMatrixFromQStringList,caption=Синтаксис]
template <class T> void THQt_ReadMatrixFromQStringList(QStringList QStringListFromFile, T **VMHL_MatrixResult);
\end{lstlisting}

\textbf{Входные параметры:}

 
QStringListFromFile --- отсюда берем информацию;

VMHL\_MatrixResult --- сюда будем записывать результат считывания матрицы.

\textbf{Возвращаемое значение:}

Отсутствует.

\textbf{Пример содержимого:}

1	2	6

52	3	96

6.4	7	4

    Второй пример содержимого:
	
1	2	6	5

52	3	96	5

-	-	4   2

Десятичные числа должны разделяться точкой.


\begin{lstlisting}[label=code_use_THQt_ReadMatrixFromQStringList,caption=Пример использования]
QString DS=QDir::separator();
QString path=QGuiApplication::applicationDirPath()+DS;//путь к папке
QStringList List = HQt_ReadFileToQStringList(path+"5.txt");

int N,M;
M=HQt_CountOfColsFromQStringList(List);
N=HQt_CountOfRowsFromQStringList(List);

double **X;
X=new double*[N];
for (int i=0;i<N;i++) X[i]=new double[M];

THQt_ReadMatrixFromQStringList(List, X);

for (int i=0;i<N;i++) delete [] X[i];
delete [] X;
\end{lstlisting}

\subsubsection{THQt\_ReadTwoVectorFromQStringList}\label{THQt_ReadTwoVectorFromQStringList}

Функция считывает данные из QStringList в два вектора.


\begin{lstlisting}[label=code_syntax_THQt_ReadTwoVectorFromQStringList,caption=Синтаксис]
template <class T> void THQt_ReadTwoVectorFromQStringList(QStringList QStringListFromFile, T *VMHL_VectorResult1, T *VMHL_VectorResult2);
template <class T> void THQt_ReadTwoVectorFromQStringList(QStringList QStringListFromFile, T *VMHL_VectorResult1, QDate *VMHL_VectorResult2);
\end{lstlisting}

\textbf{Входные параметры:}
 
QStringListFromFile --- отсюда берем информацию;
 
    VMHL\_VectorResult1 --- сюда будем записывать результат первого вектора;
 
    VMHL\_VectorResult2 --- сюда будем записывать результат второго вектора.

\textbf{Возвращаемое значение:}

Отсутствует.

\textbf{Пример содержимого:}

1	2

52	3

6.4	7

Десятичные числа должны разделяться точкой.


\begin{lstlisting}[label=code_use_THQt_ReadTwoVectorFromQStringList,caption=Пример использования]
QString DS=QDir::separator();
QString path=QGuiApplication::applicationDirPath()+DS;//путь к папке
int N;
double *x,*y;
QStringList List = HQt_ReadFileToQStringList(path+"2.txt");
N=HQt_CountOfRowsFromQStringList(List);
x=new double [N];
y=new double [N];

THQt_ReadTwoVectorFromQStringList(List,x,y);

delete [] y;
delete [] x;
\end{lstlisting}

\subsubsection{THQt\_ReadVectorFromQStringList}\label{THQt_ReadVectorFromQStringList}

Функция считывает данные из QStringList в вектор.


\begin{lstlisting}[label=code_syntax_THQt_ReadVectorFromQStringList,caption=Синтаксис]
template <class T> void THQt_ReadVectorFromQStringList(QStringList QStringListFromFile, T *VMHL_VectorResult);
\end{lstlisting}

\textbf{Входные параметры:}

QStringListFromFile --- отсюда берем информацию;

     VMHL\_VectorResult --- сюда будем записывать результат.

\textbf{Возвращаемое значение:}

Отсутствует.

\textbf{Пример содержимого:}

1

52

6.45

Десятичные числа должны разделяться точкой.


\begin{lstlisting}[label=code_use_THQt_ReadVectorFromQStringList,caption=Пример использования]
QString DS=QDir::separator();
QString path=QGuiApplication::applicationDirPath()+DS;//путь к папке
int N;
double *y;
QStringList List = HQt_ReadFileToQStringList(path+"1.txt");
N=HQt_CountOfRowsFromQStringList(List);
y=new double [N];

THQt_ReadVectorFromQStringList(List,y);//считываем

delete [] y;
\end{lstlisting}

\subsection{Получение строк-выводов по данным}

\subsubsection{HQt\_BoolToWord}\label{HQt_BoolToWord}

Функция переводит переменную, принимающую значения "0" и "1" в слова.


\begin{lstlisting}[label=code_syntax_HQt_BoolToWord,caption=Синтаксис]
QString HQt_BoolToWord(QString Bool);
QString HQt_BoolToWord(QString Bool, QString No, QString Yes);
QString HQt_BoolToWord(bool Bool);
QString HQt_BoolToWord(bool Bool, QString No, QString Yes);
QString HQt_BoolToWord(int Bool);
QString HQt_BoolToWord(int Bool, QString No, QString Yes);
\end{lstlisting}

\textbf{Входные параметры:}

Bool --- исходная переменная.
 
	No --- слово, которое означает 0 (необязательно).
 
    Yes --- слово, которое означает 1 (необязательно).

\textbf{Возвращаемое значение:}

Слово, которое характеризует переменную.


\subsubsection{HQt\_RandomString}\label{HQt_RandomString}

Функция генерирует случайную строку из английских больших и малых букв.


\begin{lstlisting}[label=code_syntax_HQt_RandomString,caption=Синтаксис]
QString HQt_RandomString(int Length);
\end{lstlisting}

\textbf{Входные параметры:}

Length --- длина строки, которую надо сгенерировать.

\textbf{Возвращаемое значение:}

Случайная строка.

\textbf{Примечание:}

Используются случайные числа, так что рекомендуется вызвать в программе иницилизатор случайных чисел qsrand.

Рекомендую так: qsrand(QDateTime::currentMSecsSinceEpoch () % 1000000);


\subsubsection{HQt\_TryingReduceString}\label{HQt_TryingReduceString}

Функция старается сократить строку длиной больше MaxSize символов, сокращая слова. Функция старается сократить строку длиной больше 40 символов, сокращая слова.


\begin{lstlisting}[label=code_syntax_HQt_TryingReduceString,caption=Синтаксис]
QString HQt_TryingReduceString(QString text, int MaxSize);
QString HQt_TryingReduceString(QString text);
\end{lstlisting}

\textbf{Входные параметры:}

text --- сокращаемая строка;
 
	MaxSize --- с какого количества символов сокращаем строку (необязательно).

\textbf{Возвращаемое значение:}

Сокращенная строка. Обратите внмиание, что строка сокращенная может быть длиннее MaxSize.


\subsubsection{HQt\_UniqueName}\label{HQt_UniqueName}

Функция возвращает уникальную строку, которую можно использовать как некий идентификатор. Собирается из "HQt\_" + текущее время или из BeginString+"\_" + текущее время.


\begin{lstlisting}[label=code_syntax_HQt_UniqueName,caption=Синтаксис]
QString HQt_UniqueName ();
QString HQt_UniqueName (QString BeginString);
\end{lstlisting}

\textbf{Входные параметры:}

BeginString --- Приставка вначале строки (необязательно).

\textbf{Возвращаемое значение:}

Уникальная строка.


\subsubsection{HQt\_UniqueNameOnlyNumbers}\label{HQt_UniqueNameOnlyNumbers}

Функция возвращает уникальную строку, которую можно использовать как некий идентификатор. В строке только цифры. Собирается из текущего времени.


\begin{lstlisting}[label=code_syntax_HQt_UniqueNameOnlyNumbers,caption=Синтаксис]
QString HQt_UniqueNameOnlyNumbers ();
\end{lstlisting}

\textbf{Входные параметры:}

Отсутствуют.

\textbf{Возвращаемое значение:}

Уникальная строка.


\subsubsection{HQt\_WriteTime}\label{HQt_WriteTime}

Функция переводит миллисекунды в строку с описанием сколько это минут, секунд и др.


\begin{lstlisting}[label=code_syntax_HQt_WriteTime,caption=Синтаксис]
QString HQt_WriteTime(int t);
QString HQt_WriteTime(qint64 t);
\end{lstlisting}

\textbf{Входные параметры:}

t --- миллисекунды.

\textbf{Возвращаемое значение:}

Строка в виде текста --- сколько секунд, минут и так далее было.


\subsection{Работа с алфавитом и переносами}

\subsubsection{HQt\_BreakdownOfTextWithWordWrap}\label{HQt_BreakdownOfTextWithWordWrap}

Функция разбивает текст на строки длиной не более length. Если может, то ставит переносы.


\begin{lstlisting}[label=code_syntax_HQt_BreakdownOfTextWithWordWrap,caption=Синтаксис]
QStringList HQt_BreakdownOfTextWithWordWrap(QString S, int length);
\end{lstlisting}

\textbf{Входные параметры:}

S --- проверяемая строка,

length ---  длина строки.

\textbf{Возвращаемое значение:}
 
Список строк, на которые разбивается текст.

\textbf{Примечание:}

Перевод слов производится по алгоритму П.Хpистова в модификации Дымченко и Ваpсанофьева.


\subsubsection{HQt\_CheckIntolerablePunctuation}\label{HQt_CheckIntolerablePunctuation}

Является ли символом знаком пунктуации, который нельзя переносить.


\begin{lstlisting}[label=code_syntax_HQt_CheckIntolerablePunctuation,caption=Синтаксис]
bool HQt_CheckIntolerablePunctuation(QString x);
\end{lstlisting}

\textbf{Входные параметры:}

x --- некий символ.

\textbf{Возвращаемое значение:}
 
true --- символ есть непереносимый символ;

false --- не из слова.


\subsubsection{HQt\_CheckLetterFromWord}\label{HQt_CheckLetterFromWord}

Является ли буква символом из слова. Считается, что это или латинские буквы, или русские, или цифры или нижнее подчеркивание. Плюc некоторые знаки препинания, так как их нельзя переносить.


\begin{lstlisting}[label=code_syntax_HQt_CheckLetterFromWord,caption=Синтаксис]
bool HQt_CheckLetterFromWord(QString x);
\end{lstlisting}

\textbf{Входные параметры:}

x --- некая буква.

\textbf{Возвращаемое значение:}
 
true --- буква из слова;

false --- не из слова.


\subsubsection{HQt\_CheckRus}\label{HQt_CheckRus}

Функция проверяет наличие русских букв в строке.


\begin{lstlisting}[label=code_syntax_HQt_CheckRus,caption=Синтаксис]
bool HQt_CheckRus(QString S);
\end{lstlisting}

\textbf{Входные параметры:}

S --- проверяемая строка.

\textbf{Возвращаемое значение:}
 
true --- естm буквы русские;
 
    false --- нет букв русских.


\subsubsection{HQt\_CutToWords}\label{HQt_CutToWords}

Функция разбивает строку на слова и те, части, между которыми они находятся. Слова с дефисом считаются за несколько слов.


\begin{lstlisting}[label=code_syntax_HQt_CutToWords,caption=Синтаксис]
QStringList HQt_CutToWords(QString S);
\end{lstlisting}

\textbf{Входные параметры:}

S - разбиваемая строка.

\textbf{Возвращаемое значение:}
 
Список подстрок, на которые можно разбить слово.

\textbf{Примечание:}

     Если кроме русского и английского языка у вас могут слова других языков, то дополните функцию HQt\_CheckLetterFromWord.


\subsubsection{HQt\_CutToWordsWithWordWrap}\label{HQt_CutToWordsWithWordWrap}

Функция разбивает строку на слова и те, части, между которыми они находятся. А русские и английские слова разделяет по переносам. Слова с дефисом считаются за несколько слов.


\begin{lstlisting}[label=code_syntax_HQt_CutToWordsWithWordWrap,caption=Синтаксис]
QStringList HQt_CutToWordsWithWordWrap(QString S);
\end{lstlisting}

\textbf{Входные параметры:}

S - разбиваемая строка.

\textbf{Возвращаемое значение:}
 
Список подстрок, на которые можно разбить слово.

\textbf{Примечание:}

     Если кроме русского и английского языка у вас могут слова других языков, то дополните функцию HQt\_CheckLetterFromWord.
	 
	 Перевод слов производится по алгоритму П. Хpистова в модификации Дымченко и Ваpсанофьева.


\subsubsection{HQt\_GetTypeCharEng}\label{HQt_GetTypeCharEng}

Функция выдает тип вводимого QString (считая, что это буква английская). Нужно для алгоритма переноса строк П.Хpистова в модификации Дымченко и Ваpсанофьева.


\begin{lstlisting}[label=code_syntax_HQt_GetTypeCharEng,caption=Синтаксис]
int HQt_GetTypeCharEng(QString x);
\end{lstlisting}

\textbf{Входные параметры:}

x --- некая буква.

\textbf{Возвращаемое значение:}

     1 --- гласная;
	 
     2 --- согласная;
	 
     0 --- иначе.


\subsubsection{HQt\_GetTypeCharRus}\label{HQt_GetTypeCharRus}

Функция выдает тип вводимого QString (считая, что это буква). Нужно для алгоритма переноса строк П.Хpистова в модификации Дымченко и Ваpсанофьева.


\begin{lstlisting}[label=code_syntax_HQt_GetTypeCharRus,caption=Синтаксис]
int HQt_GetTypeCharRus(QString x);
\end{lstlisting}

\textbf{Входные параметры:}

x --- некая буква.

\textbf{Возвращаемое значение:}

    1 --- гласная;
 
    2 --- согласная;
 
    3 --- буква из множества ьъй;
 
    0 --- иначе (английские или какие---то иные).


\subsection{Работа с датами}

\subsubsection{HQt\_DaysBetweenDates}\label{HQt_DaysBetweenDates}

Функция определяет сколько дней между двумя датами.


\begin{lstlisting}[label=code_syntax_HQt_DaysBetweenDates,caption=Синтаксис]
int HQt_DaysBetweenDates(QDate BeginDate, QDate EndDate);
int HQt_DaysBetweenDates(QString BeginDate, QString EndDate);
\end{lstlisting}

\textbf{Входные параметры:}

BeginDate --- первая дата.

EndDate --- вторая дата.

\textbf{Возвращаемое значение:}

Число дней между датами.


\subsection{Работа с файлами и папками}

\subsubsection{HQt\_CopyFile}\label{HQt_CopyFile}

Функция копирует файл filename в папку dir. Функция копирует файл filename в папку dir, с возможностью перезаписи (в функции-перегрузке).


\begin{lstlisting}[label=code_syntax_HQt_CopyFile,caption=Синтаксис]
bool HQt_CopyFile(QString filename, QString dir);
bool HQt_CopyFile(QString filename, QString dir, bool overwrite);
bool HQt_CopyFile(QString filename, QString dir, bool overwrite, bool dirmake);
\end{lstlisting}

\textbf{Входные параметры:}

filename --- имя файла (с полным путем);
 
dir --- путь к папке, куда нужно скопировать файл;

overwrite --- если true, то перезаписывать, если false, то не перезаписывать (необязательный параметр);

dirmake --- если true, то если нет директории, то она создается (необязательный параметр).

\textbf{Возвращаемое значение:}

true --- если скопировалось удачно,
 
false --- если скопировалось неудачно.


\subsubsection{HQt\_DirDelete}\label{HQt_DirDelete}

Функция удаляет директорию и всё ее содержимое.


\begin{lstlisting}[label=code_syntax_HQt_DirDelete,caption=Синтаксис]
bool HQt_DirDelete(QString path);
\end{lstlisting}

\textbf{Входные параметры:}

path --- полный путь к папке.

\textbf{Возвращаемое значение:}

true --- если удаление прошло нормально.

false ---  если удаление прошло не нормально.


\subsubsection{HQt\_DirExists}\label{HQt_DirExists}

Функция проверяет существование директории.


\begin{lstlisting}[label=code_syntax_HQt_DirExists,caption=Синтаксис]
bool HQt_DirExists(QString path);
\end{lstlisting}

\textbf{Входные параметры:}

path --- полный путь к папке.

\textbf{Возвращаемое значение:}

false --- если папка не существует;

true --- если папка существует.


\subsubsection{HQt\_DirMake}\label{HQt_DirMake}

Функция проверяет существование директории, и если ее нет, то создает.


\begin{lstlisting}[label=code_syntax_HQt_DirMake,caption=Синтаксис]
void HQt_DirMake(QString path);
\end{lstlisting}

\textbf{Входные параметры:}

path --- полный путь к папке.

\textbf{Возвращаемое значение:}

Отсутствует.


\subsubsection{HQt\_FileExists}\label{HQt_FileExists}

Функция проверяет существование файла.


\begin{lstlisting}[label=code_syntax_HQt_FileExists,caption=Синтаксис]
bool HQt_FileExists(QString filename);
\end{lstlisting}

\textbf{Входные параметры:}

filename --- имя файла.

\textbf{Возвращаемое значение:}

false --- если файл не существует;

true --- если файл существует.


\subsubsection{HQt\_GetExpFromFilename}\label{HQt_GetExpFromFilename}

Функция получает расширение файла по его имени.


\begin{lstlisting}[label=code_syntax_HQt_GetExpFromFilename,caption=Синтаксис]
QString HQt_GetExpFromFilename(QString filename);
\end{lstlisting}

\textbf{Входные параметры:}
 
filename --- имя файла.

\textbf{Возвращаемое значение:}

Строка значением расширения файла в нижнем регистре.


\subsubsection{HQt\_GetFilenameFromFullFilename}\label{HQt_GetFilenameFromFullFilename}

Функция получает имя файла по полному пути.


\begin{lstlisting}[label=code_syntax_HQt_GetFilenameFromFullFilename,caption=Синтаксис]
QString HQt_GetFilenameFromFullFilename(QString filename);
\end{lstlisting}

\textbf{Входные параметры:}
 
filename --- имя файла с путем.

\textbf{Возвращаемое значение:}

Строка с именем файла.


\subsubsection{HQt\_GetNameFromFilename}\label{HQt_GetNameFromFilename}

Функция получает имя файла без расширения по его имени.


\begin{lstlisting}[label=code_syntax_HQt_GetNameFromFilename,caption=Синтаксис]
QString HQt_GetNameFromFilename(QString filename);
\end{lstlisting}

\textbf{Входные параметры:}

filename --- имя файла (можно и с полным путем).

\textbf{Возвращаемое значение:}

Строка с именем файла без расширения.


\subsubsection{HQt\_ListDirsInDir}\label{HQt_ListDirsInDir}

Функция считывает список директорий в директории в QString.


\begin{lstlisting}[label=code_syntax_HQt_ListDirsInDir,caption=Синтаксис]
QString HQt_ListDirsInDir(QString path);
\end{lstlisting}

\textbf{Входные параметры:}

path --- путь к папке.

\textbf{Возвращаемое значение:}

 Строка со списком директорий в директории, разделенные \\n в алфавитном порядке.


\subsubsection{HQt\_ListDirsInDirQStringList}\label{HQt_ListDirsInDirQStringList}

Функция считывает список директорий в директории в QStringList..


\begin{lstlisting}[label=code_syntax_HQt_ListDirsInDirQStringList,caption=Синтаксис]
QStringList HQt_ListDirsInDirQStringList(QString path);
\end{lstlisting}

\textbf{Входные параметры:}

path --- путь к папке.

\textbf{Возвращаемое значение:}

Список строк со списком директорий в директории в алфавитном порядке.


\subsubsection{HQt\_ListFilesInDir}\label{HQt_ListFilesInDir}

Функция считывает список файлов (включая скрытые) в директории в QString.


\begin{lstlisting}[label=code_syntax_HQt_ListFilesInDir,caption=Синтаксис]
QString HQt_ListFilesInDir(QString path);
\end{lstlisting}

\textbf{Входные параметры:}

path --- путь к папке.

\textbf{Возвращаемое значение:}

Строка со списком файлов в директории, разделенные \\n в алфавитном порядке.


\subsubsection{HQt\_ListFilesInDirQStringList}\label{HQt_ListFilesInDirQStringList}

Функция считывает список файлов (включая скрытые) в директории в QStringList.


\begin{lstlisting}[label=code_syntax_HQt_ListFilesInDirQStringList,caption=Синтаксис]
QStringList HQt_ListFilesInDirQStringList(QString path);
\end{lstlisting}

\textbf{Входные параметры:}

path --- путь к папке.

\textbf{Возвращаемое значение:}

Список строк файлов в директории в алфавитном порядке.


\subsubsection{HQt\_ReadFile}\label{HQt_ReadFile}

Функция считывает текстовой файл в QString.


\begin{lstlisting}[label=code_syntax_HQt_ReadFile,caption=Синтаксис]
QString HQt_ReadFile(QString filename);
\end{lstlisting}

\textbf{Входные параметры:}

filename --- имя файла.

\textbf{Возвращаемое значение:}

Строка со всем содержимым текстового файла.


\subsubsection{HQt\_ReadFileToQStringList}\label{HQt_ReadFileToQStringList}

Функция считывает текстовой файл в QStringList.


\begin{lstlisting}[label=code_syntax_HQt_ReadFileToQStringList,caption=Синтаксис]
QStringList HQt_ReadFileToQStringList(QString filename);
\end{lstlisting}

\textbf{Входные параметры:}

filename --- имя файла.

\textbf{Возвращаемое значение:}

QStringList со всем содержимым текстового файла.


\subsubsection{HQt\_RenameFile}\label{HQt_RenameFile}

Функция переименовывает файл filename в newfilename.


\begin{lstlisting}[label=code_syntax_HQt_RenameFile,caption=Синтаксис]
bool HQt_RenameFile(QString filename, QString newfilename);
\end{lstlisting}

\textbf{Входные параметры:}

filename --- имя файла (с полным путем),
 
    newfilename --- новое имя файла (без полного пути).

\textbf{Возвращаемое значение:}

true --- если переименовалось удачно,
 
    false --- если переименовалось неудачно.


\subsubsection{HQt\_SaveFile}\label{HQt_SaveFile}

Функция сохраняет QString в текстовой файл.


\begin{lstlisting}[label=code_syntax_HQt_SaveFile,caption=Синтаксис]
void HQt_SaveFile(QString line, QString filename);
\end{lstlisting}

\textbf{Входные параметры:}

line --- содержимое, которое нужно сохранить;
 
filename --- имя файла.

\textbf{Возвращаемое значение:}

Отсутствует.


\subsection{Работа с цветом}

\subsubsection{THQt\_AlphaBlendingColorToColor}\label{THQt_AlphaBlendingColorToColor}

Функция накладывает сверху на цвет другой цвет с определенной прозрачностью.


\begin{lstlisting}[label=code_syntax_THQt_AlphaBlendingColorToColor,caption=Синтаксис]
QString THQt_AlphaBlendingColorToColor(double alpha, QString FirstRGB, QString SecondRGB);
\end{lstlisting}

\textbf{Входные параметры:}

alpha --- прозрачность второго накладываемого цвета из интервала [0;1];
 
    FirstRGB --- строка RGB кода первого цвета градиента, например: \#63ddb2;
 
    SecondRGB --- строка RGB кода последнего цвета градиента, например: \#5845da.

\textbf{Возвращаемое значение:}

Строка содержащая код цвета, например: \#25ffb5.


\subsubsection{THQt\_ColorFromGradient}\label{THQt_ColorFromGradient}

Функция выдает код RGB из градиента от одного цвета FirstRGB к другому цвету SecondRGB согласно позиции от 0 до 1.


\begin{lstlisting}[label=code_syntax_THQt_ColorFromGradient,caption=Синтаксис]
QString THQt_ColorFromGradient(double position, QString FirstRGB, QString SecondRGB);
\end{lstlisting}

\textbf{Входные параметры:}

position --- позиция из интервала [0;1], которая говорит какой цвет выдать из градиента;
 
    FirstRGB --- строка RGB кода первого цвета градиента, например: \#63ddb2;
 
    SecondRGB --- строка RGB кода последнего цвета градиента, например: \#5845da.

\textbf{Возвращаемое значение:}

Строка содержащая код цвета, например: \#25ffb5.


\subsubsection{THQt\_GiveRainbowColorRGB}\label{THQt_GiveRainbowColorRGB}

Функция выдает код RGB из градиента радуги для любой позиции от 0 до 1 из этого градиента.


\begin{lstlisting}[label=code_syntax_THQt_GiveRainbowColorRGB,caption=Синтаксис]
QString THQt_GiveRainbowColorRGB(double position);
\end{lstlisting}

\textbf{Входные параметры:}

position --- позиция из интервала [0;1], которая говорит какой цвет выдать из радуги.

\textbf{Возвращаемое значение:}

Строка содержащая код цвета, например: \#25ffb5.


\subsubsection{THQt\_RGBStringToThreeNumbers}\label{THQt_RGBStringToThreeNumbers}

Функция переводит строку RGB типа \#25ffb5 в три числа от 0 до 255, которые кодируют  цвета.


\begin{lstlisting}[label=code_syntax_THQt_RGBStringToThreeNumbers,caption=Синтаксис]
void THQt_RGBStringToThreeNumbers(QString RGB, int *R, int *G, int *B);
\end{lstlisting}

\textbf{Входные параметры:}

RGB --- строка, которая содержит код RGB цвета вида: \#25ffb5.
 
    R --- число от 0 до 255 включительно означающее красный цвет;
 
    G --- число от 0 до 255 включительно означающее зеленый цвет;
 
    B --- число от 0 до 255 включительно означающее синий цвет.

\textbf{Возвращаемое значение:}

Отсутствует.


\subsubsection{THQt\_ThreeNumbersToRGBString}\label{THQt_ThreeNumbersToRGBString}

Функция переводит три числа в строку RGB типа \#25ffb5, как в Photoshop или HTML.


\begin{lstlisting}[label=code_syntax_THQt_ThreeNumbersToRGBString,caption=Синтаксис]
QString THQt_ThreeNumbersToRGBString(int R, int G, int B);
\end{lstlisting}

\textbf{Входные параметры:}

int R --- число от 0 до 255 включительно означающее красный цвет;
 
    int G --- число от 0 до 255 включительно означающее зеленый цвет;
 
    int B --- число от 0 до 255 включительно означающее синий цвет.

\textbf{Возвращаемое значение:}

Строка содержащая код цвета, например: \#25ffb5.


\subsection{Работа со строками и списками строк}

\subsubsection{HQt\_AddUniqueQStringInQStringList}\label{HQt_AddUniqueQStringInQStringList}

Функция добавляет в QStringList строку QString. Но если такая строка уже присутствует, то добавление не происходит.


\begin{lstlisting}[label=code_syntax_HQt_AddUniqueQStringInQStringList,caption=Синтаксис]
QStringList HQt_AddUniqueQStringInQStringList (QStringList StringList, QString String);
\end{lstlisting}

\textbf{Входные параметры:}
 
StringList --- QStringList, в который мы добавляем строку (добавление в возвращаемом элементе);

     String --- добавляемая строка.

\textbf{Возвращаемое значение:}

Список строк с добавленной строкой.


\subsubsection{HQt\_IsNumeric}\label{HQt_IsNumeric}

Функция проверяет - является ли строка числом.


\begin{lstlisting}[label=code_syntax_HQt_IsNumeric,caption=Синтаксис]
bool HQt_IsNumeric(QString x);
\end{lstlisting}

\textbf{Входные параметры:}

x --- проверяемая строка.

\textbf{Возвращаемое значение:}
 
true --- является числом;

false --- не является числом.


\subsubsection{HQt\_MaxCountOfQStringList}\label{HQt_MaxCountOfQStringList}

Функция выдает длину макимальной по длине строки в QStringList.


\begin{lstlisting}[label=code_syntax_HQt_MaxCountOfQStringList,caption=Синтаксис]
int HQt_MaxCountOfQStringList(QStringList x);
\end{lstlisting}

\textbf{Входные параметры:}
 
x --- список строк.

\textbf{Возвращаемое значение:}

Длина макимальной по длине строки.


\subsubsection{HQt\_NaturalCompareTwoQStrings}\label{HQt_NaturalCompareTwoQStrings}

Функция сравнивает две строки и определяет какая строчка идет первой. Служебная функция для сортировки строк в обычном стиле, когда строки: z1, z10, z2 сортируются как z1, z2, z10.


\begin{lstlisting}[label=code_syntax_HQt_NaturalCompareTwoQStrings,caption=Синтаксис]
bool HQt_NaturalCompareTwoQStrings(const QString& s1,const QString& s2);
\end{lstlisting}

\textbf{Входные параметры:}

     s1 --- первая строка;
	 
     s2 --- вторая строка.

\textbf{Возвращаемое значение:}
 
false --- когда вторая строка должна быть первой;

true --- когда первая строка должна быть первой.


\subsubsection{HQt\_NaturalSortingQStringList}\label{HQt_NaturalSortingQStringList}

Функция сортировки строк в сортировки строк QStringList в натуральном виде, например, строки: z1, z10, z2 сортируются как z1, z2, z10.


\begin{lstlisting}[label=code_syntax_HQt_NaturalSortingQStringList,caption=Синтаксис]
QStringList HQt_NaturalSortingQStringList (QStringList StringList);
\end{lstlisting}

\textbf{Входные параметры:}

StringList --- сортируемый список строк.

\textbf{Возвращаемое значение:}
 
Отсортированный список строк.


\subsubsection{HQt\_QStringListToQString}\label{HQt_QStringListToQString}

Функция переводит QStringList в QString.


\begin{lstlisting}[label=code_syntax_HQt_QStringListToQString,caption=Синтаксис]
QString HQt_QStringListToQString(QStringList lines);
\end{lstlisting}

\textbf{Входные параметры:}
 
lines --- список строк.

\textbf{Возвращаемое значение:}

Строка с разделениями \\n.


\subsubsection{HQt\_QStringToNumber}\label{HQt_QStringToNumber}

Функция выводит строку x в число.


\begin{lstlisting}[label=code_syntax_HQt_QStringToNumber,caption=Синтаксис]
double HQt_QStringToNumber (QString x);
double HQt_QStringToNumber (QString x, bool checkcomma);
\end{lstlisting}

\textbf{Входные параметры:}

x --- строка.

checkcomma --- проверять наличие запятой (необязательно).

\textbf{Возвращаемое значение:}
 
Число из строки.


\subsubsection{HQt\_QStringToQStringList}\label{HQt_QStringToQStringList}

Функция переводит QString в QStringList.


\begin{lstlisting}[label=code_syntax_HQt_QStringToQStringList,caption=Синтаксис]
QStringList HQt_QStringToQStringList(QString line);
\end{lstlisting}

\textbf{Входные параметры:}
 
line --- строка.

\textbf{Возвращаемое значение:}

Список строк.


\subsubsection{HQt\_SearchQStringInQStringList}\label{HQt_SearchQStringInQStringList}

Функция ищет в QStringList строку QString (номер первого вхождения).


\begin{lstlisting}[label=code_syntax_HQt_SearchQStringInQStringList,caption=Синтаксис]
int HQt_SearchQStringInQStringList (QStringList StringList, QString String);
\end{lstlisting}

\textbf{Входные параметры:}
 
StringList --- QStringList, в который мы ищем строку;

String - добавляемая строка.

\textbf{Возвращаемое значение:}

Номер найденной строки. Если не найдено, то возвращается -1.


\subsubsection{HQt\_StringForLaTeX}\label{HQt_StringForLaTeX}

Функция обрабатывает строку String так, чтобы она подходила для публикации в LaTeX.


\begin{lstlisting}[label=code_syntax_HQt_StringForLaTeX,caption=Синтаксис]
QString HQt_StringForLaTeX (QString String);
\end{lstlisting}

\textbf{Входные параметры:}

String --- обрабатываемая строка.

\textbf{Возвращаемое значение:}
 
Обработанная строка.


\subsubsection{HQt\_StringToLabelForLaTeX}\label{HQt_StringToLabelForLaTeX}

Функция обрабатывает строку String так, чтобы она подходила для публикации в LaTeX в виде label.


\begin{lstlisting}[label=code_syntax_HQt_StringToLabelForLaTeX,caption=Синтаксис]
QString HQt_StringToLabelForLaTeX (QString String);
\end{lstlisting}

\textbf{Входные параметры:}

String --- обрабатываемая строка.

\textbf{Возвращаемое значение:}
 
Обработанная строка.


\subsubsection{HQt\_TextAfterEqualSign}\label{HQt_TextAfterEqualSign}

Функция возвращает текст строки после первого знака =.


\begin{lstlisting}[label=code_syntax_HQt_TextAfterEqualSign,caption=Синтаксис]
QString HQt_TextAfterEqualSign (QString String);
\end{lstlisting}

\textbf{Входные параметры:}
 
String --- строка вида: Title = Пример

\textbf{Возвращаемое значение:}

Текст строки после первого знака =.


\subsubsection{HQt\_TextBeforeEqualSign}\label{HQt_TextBeforeEqualSign}

Функция возвращает текст строки до первого знака =.


\begin{lstlisting}[label=code_syntax_HQt_TextBeforeEqualSign,caption=Синтаксис]
QString HQt_TextBeforeEqualSign (QString String);
\end{lstlisting}

\textbf{Входные параметры:}
 
String --- строка вида: Title = Пример

\textbf{Возвращаемое значение:}

Текст строки до первого знака =.


\subsubsection{THQt\_VectorToQStringList}\label{THQt_VectorToQStringList}

Функция переводит вектор чисел в QStringList.


\begin{lstlisting}[label=code_syntax_THQt_VectorToQStringList,caption=Синтаксис]
template <class T> void THQt_VectorToQStringList(T *x, int N);
\end{lstlisting}

\textbf{Входные параметры:}

x --- переводимый массив.

     N --- Количество элементов в массиве.

\textbf{Возвращаемое значение:}
 
Список строк.


\subsection{Служебные функции}

\subsubsection{HQt\_Delay}\label{HQt_Delay}

Функция делает задержку в MSecs миллисекунд.


\begin{lstlisting}[label=code_syntax_HQt_Delay,caption=Синтаксис]
void HQt_Delay(int MSecs);
\end{lstlisting}

\textbf{Входные параметры:}

MSecs --- миллисекунды, сколько надо подержать работу Qt. Не меньше пяти миллисекунд должно быть.

\textbf{Возвращаемое значение:}

Отсутствуют.

\end{document}